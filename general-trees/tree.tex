\environment ../../slide/slides

% To enable the use of sample images in ConTeXt distrubtion
\setupexternalfigures[location={local,global,default}]

\starttext

\setvariables
    [metadata]
    [
      title={General Trees},
      author={Shiv Shankar Dayal},
      date={},
      location={},
    ]

\startslide[title={Theory}]
  A {\it tree} is finite nonempty set of elements in which on element is called the {\it root} and the
  remaining elements are partitioned into $m > 0$ disjoint subsets, each of which is itself a tree. Each
  element in a tree is called a {\it node} of a tree. Each node may be the root of a tree with zero of more
  subtrees. A node with no subtrees is a {\it leaf}. We use the terms {\it father, son, brother, ancestor,
    descendant, level} and {\it depth} in the same sense as binary trees. {\bf Degree} of a node in a
  tree is the no.\ of its sons. An {\bf ordered tree} is defined as a tree in which the subtrees of each
  node form an ordered set. The first son of a node in an ordered tree is often called the {\bf oldest}
  son of that node, and the last son is called the {\bf youngest}. A {\bf forest} is an ordered set of
  ordered trees.

  Every binary tree except for the empty binary tree is indeed a tree. However, not every tree is binary. A
  tree node may have more than two sons, whereas a binary tree node may not. Even a tree whose nodes have at
  most two sons is not necessarily a binary tree. This is because an only son in a general tree is not
  designated as being a \quotation{leaf} or a \quotation{right} son, whereas in a binary tree, every son must
  be either a left son or a right son.
  \vskip 5mm
  \startC
    #define MAXS0NS 20

    struct treenode {
      int info;
      struct treenode father;
      struct treenode *sons[MXS0NS].
    }
\stopC

\startC
  // linking all nodes in a linked list
  #define MAXNOOES 500

  struct treenode {
    int info;
    int father;
    int son; // points to oldest son
    int next;
  };

  struct treenode node[MAXNODES];
\stopC
\stopslide

\startslide[title={Theory contd.}]
\startplacefigure[{location=right, none}]
  \startMPcode
    draw (-9cm, 0) -- (-9cm, 1.5cm) -- (-7cm, 1.5cm) -- (-7cm, 0) -- cycle;
    draw (-9cm, 0.5cm) -- (-7cm, 0.5cm);
    draw (-9cm, 1cm) -- (-7cm, 1cm);
    label.lft("son", (-9cm, 0.25cm));
    label.lft("info", (-9cm, 0.75cm));
    label.lft("next", (-9cm, 1.25cm));
    label("NULL", (-8cm, 1.25cm));
    label("A", (-8cm, 0.75cm));
    drawarrow (-8cm, 0.25cm) -- (-8cm, -0.5cm);
    draw (-9cm, -2cm) -- (-9cm, -0.5cm) -- (-7cm, -0.5cm) -- (-7cm, -2cm) -- cycle;
    draw (-9cm, -1.5cm) -- (-7cm, -1.5cm);
    draw (-9cm, -1cm) -- (-7cm, -1cm);
    drawarrow(-7.25cm, -.75cm) -- (-6.5cm, -.75cm);
    label("B", (-8cm, -1.25cm));
    drawarrow (-8cm, -1.75cm) -- (-8cm, -2.5cm);
    draw (-6.5cm, -2cm) -- (-6.5cm, -0.5cm) -- (-4.5cm, -0.5cm) -- (-4.5cm, -2cm) -- cycle;
    draw (-6.5cm, -1.5cm) -- (-4.5cm, -1.5cm);
    draw (-6.5cm, -1cm) -- (-4.5cm, -1cm);
    label("C", (-5.5cm, -1.25cm));
    label("NULL", (-5.5cm, -1.75cm));
    draw (-9cm, -4cm) -- (-9cm, -2.5cm) -- (-7cm, -2.5cm) -- (-7cm, -4cm) -- cycle;
    draw (-9cm, -3.5cm) -- (-7cm, -3.5cm);
    draw (-9cm, -3cm) -- (-7cm, -3cm);
    label("E", (-8cm, -3.25cm));
    label("NULL", (-8cm, -3.75cm));
    draw(-4cm, -2cm) -- (-4cm, -0.5cm) -- (-2cm, -0.5cm) -- (-2cm, -2cm) -- cycle;
    draw (-4cm, -1.5cm) -- (-2cm, -1.5cm);
    draw (-4cm, -1cm) -- (-2cm, -1cm);
    label("D", (-3cm, -1.25cm));
    label("NULL", (-3cm, -.75cm));
    drawarrow (-4.75cm, -.75cm) -- (-4cm, -.75cm);
    draw (-6.5cm, -4cm) -- (-6.5cm, -2.5cm) -- (-4.5cm, -2.5cm) -- (-4.5cm, -4cm) -- cycle;
    draw (-6.5cm, -3.5cm) -- (-4.5cm, -3.5cm);
    draw (-6.5cm, -3cm) -- (-4.5cm, -3cm);
    label("F", (-5.5cm, -3.25cm));
    label("NULL", (-5.5cm, -3.75cm));
    label("NULL", (-5.5cm, -2.75cm));
    drawarrow (-7.25cm, -2.75cm) -- (-6.5cm, -2.75cm);
    draw (-4cm, -4cm) -- (-4cm, -2.5cm) -- (-2cm, -2.5cm) -- (-2cm, -4cm) -- cycle;
    draw (-4cm, -3.5cm) -- (-2cm, -3.5cm);
    draw (-4cm, -3cm) -- (-2cm, -3cm);
    label("G", (-3cm, -3.25cm));
    label("NULL", (-3cm, -3.75cm));
    label("NULL", (-3cm, -2.75cm));
    drawarrow (-3cm, -1.75cm) -- (-3cm, -2.5cm);
  \stopMPcode
\stopplacefigure
  \startC
  struct treenode {
    int info;
    struct treenode *father.
    struct treenode *son;
    struct treenode *next.
  };

  typedef struct treenode *NODEPTR;
\stopC

Like binary trees if we traverse from root to sons then father field is not needed. Given below is the code
for inorder traversal of such a tree.
\vskip 5mm
\startC
  void inorder_trav(NODEPTR root) {
    if(root != NULL) {
      inorder_trav(root->son);
      printf("%d\n", root->info);
      inorder_trav(root->next);
    }
  }
\stopC
\vskip 5mm
Similarly we can draw inspiration from BST traversal and implement preorder and postorder traversal for such
trees.
\stopslide
\startslide[title={Constructing a Tree}]
  We assume that father node is not needed.
  \vskip 5mm
  \startC
  // p is a pointer to a node and list is a linear list of nodes linked through the next fields.
  void set_sons(NODEPTR p, NODEPTR list) {
    if(p == NULL) {
      printf("Invalid ndoe\n");
      exit(1);
    }

    if(p->son != NULL) {
      printf("p must have a son.\n");
      exit(2);
    }

    p->son = list;
  }
\stopC
\stopslide
\startslide[title={Constructing a Tree}]
  Now we will add a son to a list as youngest son.
  \vskip 5mm
  \startC
    void addson(NODEPTR p, int x) {
      NODEPTR q;

      if(p == NULL) {
        printf("Invalid ndoe\n");
        exit(1);
      }

      NODEPTR r = NULL;
      q = p->son;

      while(q != NULL) {
        r = q;
        q = q->next;
      }

      q = (void*)malloc(sizeof(NODEPTR));
      q->info = x;
      q->next = NULL

      if(r == NULL)
        p->son = q;
      else
        r->next = q;
    }
  \stopC
\stopslide
\stoptext
