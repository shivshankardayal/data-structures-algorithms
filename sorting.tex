\documentclass[aspectratio=169,8pt]{beamer}

% Standard packages

\usepackage[english]{babel}
%\usepackage[latin1]{inputenc}
%\usepackage{times}
%\usepackage[T1]{fontenc}
\usepackage{fontspec}
\usepackage[]{unicode-math}
\setmathfont{Inconsolata}
\setsansfont{Roboto}


% Setup TikZ

\usepackage{tikz}
\usetikzlibrary{arrows}
\tikzstyle{block}=[draw opacity=0.7,line width=1.4cm]

\newcounter{counter}
% Author, Title, etc.

\title{Complex Numbers}

\author[Shiv Shankar Dayal]{Shiv Shankar Dayal}

\begin{document}
\begin{frame}
       \titlepage
\end{frame}
\begin{frame}{Theory}
  A complex number comprises of two numbers: a real number and an imaginary number. An imaginary number is square root of a
  negative number, for example, $\sqrt{-1}, \sqrt{-2}, \sqrt{-3}.$ These are called imaginary numbers because they do not exist in
  real life in the sense that like ordinary numbers they cannot be used for counting.\\
  \vspace*{0.2cm}
  A real number like $1$ can also be represented as a complex number having a $0$ imaginary part. The value $\sqrt{-1}$ is denoted
  by the Greek letter $\iota,$ which stands for \textit{iota.} Typically, we use either $i$ or $j$ to denote this.\\
  \vspace*{0.2cm}
  Clearly we have following:
  $$i^2 = -1, i^3 = -i, i^4 = 1, i^5 = i, i^6 = -1, i^7 = -i, i^8 = 1, \ldots$$
  If you examine carefully you will find that following holds true
  $$i^{4m} = 1, i^{4m + 1} = i, i^{4m + 2} = -1\text{~and~}i^{4m + 3}= -i~\forall~m\in P$$
  $P$ is the set of positive integers including zero.

  \textbf{Note:} $1 = \sqrt{1} = \sqrt{-1*-1} = i * i = -1$

  However, the above result is wrong because for any two real numbers $a$ and $b$ the result $\sqrt{a}*\sqrt{b} = \sqrt{ab}$ holds
  good if and only if the two numbers are zero or positive. Thus $1 = \sqrt{-1*-1}$ is wrong because power of $-$ is $-1$ which
  makes the set of equalities go wrong.
\end{frame}
\begin{frame}{Definitions}
  A complex number is commonly written as $a + ib$ or $x + iy.$ Here $a, b, x$ and $y$ are all real numbers. The complex number
  itself is denoted by $z,$ like $z = x + iy.$ Here $x$ is called the \textit{real} part and is also denote by $Re(z)$ and $y$ is
  called the imaginary part and is also denoted by $Im(z).$\\
  \vspace*{0.2cm}
  A complex number is purely real if its imaginary part or $y$ or $Im(z)$ is zero. Similarly, a complex number is purely imaginary
  if its real part or $x$ or $Re(z)$ is zero. Clearly, as you can fathom that there can exist only one number which has both the
  parts as zero and certainly that is $0.$ That is, $0=0+i0.$\\
  \vspace*{0.2cm}
  The set of all complex number is typically denoted by $C$. Two complex numbers $z_1$ and $z_2$ are said to be true if there real
  parts are equal and imaginary parts are equal. That is if $z_1 = x_1 + iy_1$ and $z_2 = x_2 + iy_2$ then for $z_11$ to be equal
  to $z_2,$ $x_1$ must be equal to $x_2$ and $y_1$ must be equal to $y_2.$
\end{frame}
\begin{frame}{Simple Operations}
  \begin{enumerate}
  \item \textbf{Addition:} $(a + ib) + (c + id) = (a + c) + i(b + d)$
  \item \textbf{Subtraction:} $(a + ib) - (c + id) = (a - c) + i(b - d)$
  \item \textbf{Multiplication:} $(a + ib) * (c + id) = ac + ibc + iad + bdi^2 = (ac - bd) + i(bc + ad)$
    \item \textbf{Division:} $\frac{a + ib}{c + id} = \frac{a + ib}{c + id}.\frac{c - id}{c - id} = \frac{ac + bd + i(bc + ad)}{c^2
    + d^2}$
  \end{enumerate}
  \vspace*{.2cm}
  \textbf{\large{Conjugate of a Complex Number}}\\
  \vspace*{.2cm}
  Let $z = x + iy$ be a complex number then its complex conjugate is a number with imaginary part made negative and it is written
  as $\overline{z} = x - iy.$ $\overline{z}$ is the typical representation for a conjugate of a complex number $z$.\\
  \vspace*{0.2cm}
  \textbf{Properties of Conjugates}\\
  \vspace*{0.2cm}
  \begin{enumerate}
  \item $z_1 = z_2 \Leftrightarrow \overline{z_1} = \overline{z_2}$\\
    Clearly as we know for two complex numbers to be equal both parts must be equal so this is very easy to understand that if $x_1
    = x_2$ and $y_1 = y_2$ then this bidirectional condition is always satisfied.
  \item $\overline(\overline{z}) = z.$\\
    $z = x + iy,$ hence, $\overline{z} = x - iy,$ hence $\overline(\overline{z}) = x - (-iy) = x + iy = z$
  \item $z + \overline{z} = 2Re(x)$\\
    Clearly, $z + \overline{z} = x + iy + x - iy = 2x = 2Re(x)$
  \item $z - \overline{z} = 2iIm(x)$\\
    Clearly, $z - \overline{z} = x + iy - (x - iy) = 2iy = 2iIm(x)$
    \setcounter{counter}{\value{enumi}}
  \end{enumerate}
\end{frame}
\begin{frame}{Conjugate contd.}
  \begin{enumerate}
    \setcounter{enumi}{\value{counter}}
  \item $z + \overline{z} = 0 \Leftrightarrow z$ is purely imaginary.\\
    $z + \overline{z} = x + iy + x - iy = 2x = 0$ which means rela part is zero and hence $z$ is purely imaginary.
  \item $z = \overline{z} \Leftrightarrow z$ is purely real.\\
    $x + iy = x -iy \Rightarrow 2iy = 0$ and thus $z$ is purely real.
  \item $z\overline{z} = [x^2 + y^2]$\\
    Clearly, $z\overline{z} = (x + iy)(x - iy) = x^2 + y^2$
  \item $\overline{z_1 + z_2} = \overline{z_1} + \overline{z_1}\overline{z_1 + z_2} = \overline{(x_1 + iy_1) + (x_2 + iy_2)} = \overline{(x_1 + x_2) + i(y_1 + y_2)}$\\
    $= (x_1 + x_2) -i(y_1 + y_2) = x_1 - iy_1 + x_2 - iy_2 = \overline{z_1} + \overline{z_2}$
  \item $\overline{z_1 - z_2} = \overline{z_1} - \overline{z_2}$\\
    It can be proven like item 8.
  \item $\overline{z_1z_2} = \overline{z_1} - \overline{z_2}$\\
    It can be proven like item 8.
  \item $\overline{\left(\frac{z_1}{z_2}\right)} = \frac{\overline{z_1}}{\overline{z_2}}$ if $z_2\neq 0$
    You can rationalize the base by multiplying it from its conjugate and apply division formula given above to prove it.
  \item If $P(z) = a_0 + a_1z + a_2z^2 + \ldots + a_nz^n.$ where $s_0, a_1, \ldots, a_n$ and $z$ are complex numbers, then\\
    $\overline{P(z)} = \overline{a_0} + \overline{a_1}\overline{z} + \overline{a_2}(\overline{z})^2 + \ldots + \overline{a_n}(\overline{z)^n} = \overline{P}(\overline{z})$ where\\
    $\overline{P}(z) = \overline{a_0} + \overline{a_1}z + \overline{a_2}z^2 + \ldots + \overline{a_n}z^n$
    \setcounter{counter}{\value{enumi}}
  \end{enumerate}
\end{frame}
\begin{frame}{Conjugate contd.}
  \begin{enumerate}
    \setcounter{enumi}{\value{counter}}
  \item If $R(z) = \frac{P(z)}{Q(z)}$ where $P(z)$ and $Q(z)$ are polynomilas in $z,$ and $Q(z)\neq 0,$ then\\
    $\overline{R(z)} = \frac{\overline{P}(\overline{z})}{\overline{Q}(\overline{z})}$
  \item If $z = \begin{vmatrix}a_1 & a_2 & a_3\\b_1 & b_2 & b_3\\c_1 & c_2 & c_3\end{vmatrix}$, then $\overline{z} = \begin{vmatrix}\overline{a_1} & \overline{a_2} & \overline{a_3}\\\overline{b_1} & \overline{b_2} & \overline{b_3}\\\overline{c_1} & \overline{c_2} & \overline{c_3}\end{vmatrix}$ where $a_i, b_i, c_i(i = 1,2,3)$ are complex numbers.
  \end{enumerate}
  \vspace*{0.2cm}
  \textbf{\large{Modulus of a Complex Number}}\\
  \vspace*{0.2cm}
  Modulus of a complex numbe $z$ is denoted by $|z|$ and is equalt to the real number $\sqrt{x^2 + y^2}$. Note that $|z|\geq 0~\forall~z\in C$\\
  \vspace*{0.2cm}
  \textbf{Properties of Modulus}\\
  \vspace*{0.2cm}
  \begin{enumerate}
  \item $|z| = 0 \Leftrightarrow z = 0.$\\
    $x^2 + y^2 = 0 \Leftrightarrow x = 0, y = 0 \Rightarrow z = 0$
  \item $|z| = |\overline{z}| = |-z| = |-\overline{z}| = x^2 + y^2$
  \item $-|z|\leq Re(x)\leq |z|$ Clearly, $-(x^2 + y^2) \leq x^2 \leq (x^2 + y^2)$
  \item $-|z|\leq Im(x)\leq |z|$ Clearly, $-(x^2 + y^2) \leq y^2 \leq (x^2 + y^2)$
  \item $z\overline{z} = |z|^2$ Clearly, $(x + iy)(x - iy) = (x^2 + y^2) = |z|^2$
  \item $|z_1z_2| = |z_1||z_2|$ Clearly, $|z_1z_2| = |x_1x_2 - y_1y_2 + i(x_1y_2 + x_2y_1))|$\\
    $= \sqrt{(x_1x_2 - y_1y_2)^2 + (x_1y_2 + x_2y_1)^2} = \sqrt{(x_1^2 + y_1^2)(x_2^2 + y_2^2)} = |z_1||z_2|$
    \setcounter{enumi}{\value{counter}}
  \end{enumerate}
\end{frame}
\begin{frame}{Modulus contd.}
  \begin{enumerate}
    \setcounter{enumi}{\value{counter}}
  \item $\left|\frac{z_1}{z_2}\right| = \frac{|z_1|}{z_2},$ if $z_2\neq 0$
  \item $|z_1 + z_2|^2 = |z_1|^2 + |z_2|^2 + \overline{z_1}z_2 + z_1\overline{z_2} = |z_1|^2 + |z_2|^2 + 2Re(z_1\overline{z_2})$
  \item $|z_1 - z_2|^2 = |z_1|^2 + |z_2|^2 - \overline{z_1}z_2 - z_1\overline{z_2} = |z_1|^2 + |z_2|^2 - 2Re(z_1\overline{z_2})$
  \item $|z_1 + z_2|^2 + |z_1 - z_2|^2 = 2(|z_1|^2 + |z_2|^2)$
  \item If $a$ amd $b$ are real numbers and $z_1$ and $z_2$ are complex numbers, then\\
    $|az_1 + bz_2|^2 + |bz_1 - az_2|^2 = (a^2 + b^2)(|z_1|^2 + |z_2|^2)$
  \item If $z_1, z_2\neq 0,$ then $|z_1 + z_2|^2 = |z_1|^2 + |z_2|^2 \Leftrightarrow \frac{z_1}{z_2}$ is purely imaginary.
  \item If $z_1$ and $z_2$ are complex numbers then $|z_1 + z_2|\leq |z_2| + |z_2|.$ This expression can be generalized to $n$ terms as well.
  \item Simialrly, these can be proven that $|z_1 - z_2|\leq |z_1| + |z_2|, |z_1| - |z_2|\leq |z_1| + |z_2|$ and $|z_1 - z_2|\geq ||z_1| - |z_2||$
  \end{enumerate}
\end{frame}
\begin{frame}{Theory contd}
  A complex number $z$ which we have considered to be equal to $x + iy$ can be
  represented by a point $P$ whose caretesian coordinates are $(x, y)$ referred
  to rectangular axes $Ox$ and $Oy$ where $O$ is origin i.e. $(0, 0)$ and are
  called \textit{real} and \textit{imaginary} axis respectively. The $xy$ two
  dimensional plane is also called \textit{Argand plane, complex plane or
    Gaussian plane}. The point $P$ is also called the \textit{image} of the
  complex number and $z$ is also called the \textit{affix} or \textit{complex
    coordinate} of point $P.$

  The modulus is given by the length of segment $OP$ which is equal to $OP =
  \sqrt{x^2 + y^2} = |z|.$ Thus, $|z|$ is the length of $OP$.
\end{frame}
\begin{frame}
  \begin{center}
    \begin{tikzpicture}
      \draw[->] (-.5,0) -- (3,0);
      \draw[->] (0,-.5) -- (0,3);
      \draw (0, 3.5) node {$Y$};
      \draw (3.5, 0) node {$X$};
      \draw (2.5,0) -- (2.5,2.5);
      \draw (0,0) -- (2.5, 2.5);
      \draw (.5,0) arc(0:45:.5);
      \draw (.7,.3) node{$\theta$};
      \draw (1.5, 0.2) node {$x$};
      \draw (2.7, 1.5) node {$y$};
      \draw (3.4, 1.1) node {$OP=|z|$};
      \draw (3.5, 0.7) node {$arg(z)=\theta$};
      \draw (2.5, 2.7) node{$P=x+iy$};
      \node [label = below left:{$O$}] (o) at (0, 0) {};
      \draw (o);
    \end{tikzpicture}
  \end{center}
  In the diagram $\theta$ is known as the argument of $z.$ it is the angle made
  with positive direction(i.e. counter-clockwise) of real axis. This arhument
  is not unique. If $\theta$ is an argument of a complex number $z$ then $2n\pi
  + \theta$ where $n\in I$ where $I$ is the set of integers will be arguments
  as well. The value of argument for which $-\pi<\theta\leq \pi$ is called the
  \textit{principal argument.}
\end{frame}
\begin{frame}{Different Arguments of a Complex Number}
  In the digram given in previous slide the argument is given as
  $$arg(z) = \tan^{-1}\left(\frac{y}{x}\right)$$
  this value is for when $z$ is in first quadrant. When $z$ will lie in second,
  third and fourth quadrants then arguments will be
  $$arg(z) = \pi - \tan^{-1}\left(\frac{y}{|x|}\right), arg(z) = -\pi +
  \tan^{-1}\left(\frac{|y|}{|x|}\right), arg(z) =
  -\tan^{-1}\left(\frac{|y|}{x}\right)$$
  \vspace*{0.2cm}
  \textbf{\large{Polar Form of a Complex Number}}\\
  \vspace*{0.2cm}
  If $z$ is a non-zero complex number, then we can write $z = r(\cos\theta +
  i\sin\theta)$ where $r = |z|$ and $\theta = arg(z)$\\
  \vspace*{0.2cm}
  In this case $z$ is also given by $z = r[\cos(2n\pi + \theta) + i\sin(2n\pi +
    \theta)]$ where $n\in I.$\\
  \vspace*{0.2cm}
  \textbf{\large{Euler's Formula}}\\
  The complex number $\cos\theta + i\sin\theta$ is denoted by $e^{i\theta}$.
\end{frame}
\begin{frame}{Properties of Arguments}
  If $z, z_1$ and $z_2$ are complex numbers then
  \begin{enumerate}
    \item $arg(\overline{z}) = -arg(z)$. This can be easily proven as $z = x +
      iy$ and $\overline{z} = x - iy$ so sign of argument will get a -ve sign
      as $y$ gets one.
    \item $arg(z_1z_2) = arg(z_1) + arg(z_2) + 2k\pi$ where
      $$k = \begin{cases}0 & -\pi <arg(z_1) + arg(z_2) \leq \pi\\
      1 & -2\pi < arg(z_1) + arg(z_2)\leq -\pi\\
      -1 & -\pi < arg(z_1) + arg(z_2)\leq 2\pi\end{cases}$$
    \item $arg(z_1\overline{z_2}) = arg(z_1) - arg(z_2)$
    \item $\arg\left(\frac{z_1}{z_2}\right) = arg(z_1) + arg(z_2) + 2k\pi$
      where $k$ is same as item 2 with $+$ sign between $z_1$ and $z_2$ are
      replaced with $-$ sign.
    \item $|z_1 + z_2| = |z_1 - z_2|\Leftrightarrow arg(z_1) - arg(z_2) =
      \pi/2$
    \item $|z_1 + z_2| = |z_1| + |z_2|\Leftrightarrow arg(z_1) = arg(z_2)$
    \item $|z_1 + z_2|^2 = r_1^2 + r_2^2 + 2r_1r_2\cos(\theta_1 - \theta_2)$
    \item $|z_1 - z_2|^2 = r_1^2 + r_2^2 + 2r_1r_2\cos(\theta_1 + \theta_2)$
  \end{enumerate}
\end{frame}
\begin{frame}{Vector Representation}
  Complex numbers can also be represented as vectors. Length of the vector is
  nothing but modulus of complex number and argument is the angle which the
  vector makes with the real axis. It is denoted as $\overrightarrow{OP}$ where
  $OP$ represents the vector of the complex number $z.$
\end{frame}
\end{document}
