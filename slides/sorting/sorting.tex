\documentclass[aspectratio=169,8pt]{beamer}

% Standard packages

\usepackage[english]{babel}
%\usepackage[latin1]{inputenc}
%\usepackage{times}
%\usepackage[T1]{fontenc}
\usepackage{fontspec}
\usepackage[]{unicode-math}
\setmathfont{Inconsolata}
\setsansfont{Roboto}


% Setup TikZ

\usepackage{tikz}
\usetikzlibrary{arrows}
\tikzstyle{block}=[draw opacity=0.7,line width=1.4cm]

\newcounter{counter}
% Author, Title, etc.

\title{Sorting}

\author[Shiv Shankar Dayal]{Shiv Shankar Dayal}

\begin{document}
\begin{frame}
  \titlepage
\end{frame}
\begin{frame}{Sorting}
  Sorting is the process of arranging a list of values in increasing/decreasing order.\\
  \vspace*{0.2cm}
  Some algorithms(for example, insertion sort or selection sort) do not require extra space/memory to sort the list of values while
  some algorithms(merge sort and its variants) require extra space for sorting the list. Alogrithms which do not require extra
  space are called \textit{in-place} sorting algorithm.\\
  \vspace*{0.2cm}
  Internal/External Sorting: When all the data is stored in memory the sorting is known as internal sorting. When data does not fit
  in memory and needs to use secondary storage or disk then it is called external sorting. Note that in modern times we have so
  much memory that we typically never sort externally making this kind of algorithms obsolete and are of academic interest only.\\
  \vspace*{0.2cm}
  Stability of Sorting Algorithms: A sorting algorithm in which values with equal keys appear in the same order in sorted output as
  they appear in the input array to be sorted is called stable sorting algorithm. For example, if we have tuples of students with
  marks and names then if we sort by name or marks then we might lose the original order after sorting. Note that stability won't
  be a concern if all marks and names are different i.e. all keys are different. Unstable sorts can be made stable by modifying the
  key comparison operation.\\
  \vspace*{0.2cm}
  Mechanism of Sorting: There are three ways sorting algorithms work. By exchange, by insetion and by selection. We will see what
  these mean when we study them. There are other tecniques also like enumeration and special-purpose sorting as well.
\end{frame}
\end{document}
