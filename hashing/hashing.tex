% -*- mode: latex; -*-
\DocumentMetadata{pdfstandard=X-4p}
\documentclass[aspectratio=169,8pt]{beamer}
\usetheme{metropolis}

% Standard packages

\usepackage[english]{babel}
%\usepackage[latin1]{inputenc}
%\usepackage{times}
%\usepackage[T1]{fontenc}
\usepackage{fontspec}
\usepackage[]{unicode-math}
\setsansfont{Roboto}
\usepackage{amsmath}

% Setup asymptote
\usepackage[inline]{asymptote}

\newcounter{counter}
% Author, Title, etc.

\title{Hashing}

\author[Shiv Shankar Dayal]{Shiv Shankar Dayal}

\begin{document}
\begin{frame}
  \titlepage
\end{frame}
\begin{frame}{Introduction to Hashing}
  Hashing is a technique to compute some value from a given input. This value can be of fixed length or
  variable length. We can treat this as a mathematical function, which is called a {\it hash function}. The
  output of these hash functions is called {\it keys}, the input is called {\it value} and the table where
  keys and values are stored is called the hash table. The main purpose of a hash function is to perform a
  fast search. The trees which we have studied are at best $O(\log N)$.

  Consider most common words of English. These are A, AND, ARE, AS, AT, BE, BUT, BY, FOR, FROM,  HIS, IN,
  IS, IT, ON, OR, OF, ON, OR, WAS, WHICH, WITH, HAD, HAVE, HE, HER, THAT, THE and THIS. Let us consider a
  function $f(x)$, which can transform these 29 words between $10$ and $28$. We will need about 39 table
  locations to store the 29 keys.

  However, it is very hard to find a good hash function. There are about $39^{29}=10^{46}$ possible
  functions from a $29$-value into a $39$-value set and only $39!/10!=5.6\times10^{39}$ of them will give
  distinct values for each argument.

  You can study birthday paradox in details which says that if $23$ people are present in a room then there
  is bound to be two people having same birthday. If we select a random function that maps $23$ keys into a
  table of size $365$, the probability that no two keys map to same location is only $.4927$.
\end{frame}
\begin{frame}{Contd.}
  The duplicate values are called collisions. The problem of finding a good hash function is so severe that
  GCC uses red-black trees as storage medium for C++ map/hash table and the argument which is put forth by
  GCC documentation is that with collisions most hash functions worsen to $O(\log N)$ complexity.
\end{frame}
\begin{frame}{Properties of a good hash function}
  Following are desirable qualities of a hash function:
  \begin{itemize}
  \item Efficient to compute so that insertion in hash table is easy,
  \item Keys should be uniformly distributed,
  \item Minimum collision i.e. high periodicity(when different inputs produce same value then it is called a
    collision), and
  \item Number of items in the table divided by the size of the table i.e. load factor should be low.
  \end{itemize}
\end{frame}
\begin{frame}{Some simple hash functions}
  \begin{itemize}
    \item {\bf Division Method:} $f(x) = x\mod M$, here $x$ is the input value and $M$ is the hash table
      size. A prime $M$ is suitable as that makes sure that distribution is even. However, it leads to poor
      performance as consecutive keys map to consecutive locations.
    \item {\bf Middle Sqaure Method:} In this case we square the key $x$ and extract $m$ middle digits from
      the square. The benefit is that result is not dominated by distribution of top and bottom
      digits. However, the size of key and collisions are limitations are the problem.
    \item {\bf Digit Folding Method:} In this method, we divide the key $x$ into parts $x_1, x_2, \ldots,
      x_n$, where each part has the same number of digits except for the last part that can have less digits
      than the other parts; finally we add the parts.
    \item {\bf Multiplication Method:} We choose a constant value $A\in(0, 1)$, multiply key with $A$,
      extract the fractional part, multiply the result with $M$, and take the floor of result.
  \end{itemize}
\end{frame}
\begin{frame}{Collision resolution techniques}
  \begin{itemize}
  \item Separate chaining/Open hashing
  \item Open addressing/Closed hashing
    \begin{itemize}
    \item Linear probing
    \item Quadratic probing
    \item Double hashing
    \end{itemize}
  \end{itemize}
\end{frame}
\begin{frame}{Separate chaning/Open hashing}
  Each cell of the hash table points to a linked list of records that have the same hash
  function value. Hash function is $x \mod 5$. Input values are $12, 16, 22, 25, 35, 45, 63$.

  $|25->35->45|16|12 -> 22|63|||$
\end{frame}
\begin{frame}{Linear probing}
  Next available slot is taken in case of collision. Hash function is $x \mod 5$. Input values are $12, 16,
  22, 25, 35$.

  $|25|16|12|22|35|$
\end{frame}
\begin{frame}{Quadratic probing}
  If the slot $hash(x)\%M$ is full then try $hash(x) + 1^2$, next try $hash(x) + 2^2$, next try $hash(x) +
  3^3$ and so on.
\end{frame}
\begin{frame}{Double hashing}
  $h(x, y) = (h1(x) + y * h2(x)) \% M$
\end{frame}
\end{document}
